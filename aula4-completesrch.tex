%- - - - - - - - - - - - - - - - - - - - - - - - - - - - - - - - - SLIDE -
\begin{frame}
\frametitle{Complete Search}
\begin{block}{}
\begin{itemize}
	\bitem Estratégia baseada no princípio KISS (``Keep It Simple, Stupid'')
	\begin{itemize}
		\bitem Buscar os resultados evitando qualquer complexidade desnecessária.
	\end{itemize}
	\bitem O objetivo numa competição de programação é escrever um programa que resolva o problema dentro do tempo limite.
	\begin{itemize}
		\bitem Não importa se existe ou não uma solução mais eficiente.
	\end{itemize}
	\bitem A busca completa faz uso do método trivial, de força bruta, todas as possíveis soluções são analisadas para encontrar a resposta.
	\bitem Essa técnica sempre deve ser a primeira a ser considerada.
	\begin{itemize}
		\bitem Caso funcione dentro do limite de tempo / memória, use-a! 
		\begin{itemize}
			\bitem Geralmente é fácil de codificar e debugar.
%			\bitem Mais tempo para trabalhar nos problemas difíceis. 
		\end{itemize}
	\end{itemize}
	\bitem Apenas alguns milhões de possíveis respostas para um problema? Itere em todas elas e encontre aquela que funciona.
\end{itemize}
\end{block}
\pause

\begin{block}{\tiny Cuidado}
Nem sempre é óbvio que a busca completa pode ser usada.
\end{block}
\end{frame}

%- - - - - - - - - - - - - - - - - - - - - - - - - - - - - - - - - SLIDE -
\begin{frame}
\frametitle{Complete Search}

\begin{block}{}
\begin{itemize}
	\bitem Uma das técnicas de resolução de problemas mais importantes;
	\begin{itemize}
		\bitem Pode ser aplicada a uma grande gama de problemas quando as instâncias são pequenas os suficiente
		\bitem Ponto de partida para o desenvolvimento de outros algoritmos.
	\end{itemize}
	\bitem Competidor precisa saber:
	\begin{itemize}
		\bitem Gerar/testar: subconjuntos, permutações, ...
		\bitem Técnicas para reduzir o espaço de busca
		\bitem Estimar a complexidade no pior caso
	\end{itemize}
	\bitem Ajustes no código podem influenciar bastante no tempo de execução;
	\begin{itemize}
		\bitem Vale a pena implementar a mesma solução de formas diferentes.
	\end{itemize}
\end{itemize}	
\end{block}

\begin{block}{\tiny \#protip}
Quando não conseguir pensar em um jeito melhor de resolver o problema arrisque a solução por força bruta.
Caso exista um caso onde ela não é rápida o suficiente, mantenha a solução por perto e use-a para testar outras soluções nos casos menores.
\end{block}
\end{frame}

%- - - - - - - - - - - - - - - - - - - - - - - - - - - - - - - - - SLIDE -
%    - Filtrar vs. Gerar    
%    - busca em largura vs. busca em profundidade

%- - - - - - - - - - - - - - - - - - - - - - - - - - - - - - - - - SLIDE -
%    Selecionar alguns problemas para discutir ...

%- - - - - - - - - - - - - - - - - - - - - - - - - - - - - - - - - SLIDE -
%Dicas:
%    - Podar o quanto antes
%    - Aproveitar simetrias
%    - Pré-cálculo
%    - inverter o problema (?)
%    - usar um algoritmo/estrutura de dados melhor
%    - Otimizar código

%- - - - - - - - - - - - - - - - - - - - - - - - - - - - - - - - - SLIDE -
\begin{frame}
\frametitle{Leituras Recomendadas}

\begin{block}{}
\begin{itemize}
\tiny
	\bitem \url{http://community.topcoder.com/tc?module=Static&d1=tutorials&d2=recursionPt1}
	\bitem \url{http://community.topcoder.com/tc?module=Static&d1=tutorials&d2=recursionPt2}
	\bitem \url{http://www.inf.ufg.br/~paulocosta/tap/material/bt1.pdf}
	\bitem \url{http://www.inf.ufg.br/~paulocosta/tap/material/bt2.pdf}
%	\bitem \url{http://www.inf.ufg.br/~paulocosta/tap/material/bt3.pdf} % MITM
	\bitem \url{http://www.comp.nus.edu.sg/~stevenha/visualization/recursion.html}
\end{itemize}
\end{block}

\end{frame}
