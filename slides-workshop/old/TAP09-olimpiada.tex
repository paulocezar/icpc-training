%- - - - - - - - - - - - - - - - - - - - - - - - - - - - - - - - - SLIDE -
\begin{frame}
 \frametitle{Competições de Programação}
 \begin{block}{\em International Olympiad in Informatics}
  \begin{overprint}
% .................................
   \onslide<1| handout:1>
   \begin{itemize}
    \item Idéia nasceu no congresso da UNESCO em 1987 na Bulgária. A primeira IOI ocorreu em 1989, também na Bulgária.
    \item O objetivo principal é estimular o interesse em Ciência da Computação, além de trazer talentos na área de todos os países do mundo que podem assim compartilhar experiências culturais e científicas.
    \item É uma das seis olimpíadas científicas reconhecidas.
    \item \url{http://en.wikipedia.org/wiki/International_Olympiad_in_Informatics}
   \end{itemize}
% .................................
   \onslide<2| handout:2>
   \begin{description}
    \item [2005] Novy Sacz, Polônia: \url{http://www.ioi2005.pl}
    \item [2006] Mérida, México: \url{http://www.ioi2006.org}
    \item [2007] Zagreb, Croácia: \url{http://ioi2007.hsin.hr}
    \item [2008] Cairo, Egito: \url{http://www.ioi2008.org}
    \item [2009] Plovdiv, Bulgária: \url{http://www.ioi2009.org}
    \item [2010] Waterloo, Ontario, Canada
    \item [2011] Tailândia
    \item [2012] Itália
   \end{description}
% .................................
  \end{overprint}
 \end{block}
\end{frame}
%- - - - - - - - - - - - - - - - - - - - - - - - - - - - - - - - - SLIDE -
\begin{frame}
 \frametitle{Competições de Programação}
 \begin{block}{Olimpíada Brasileira de Informática}
  \begin{itemize}
   \item OBI ocorre desde o ano de 1999.
   \item Evento da SBC desde seu início, coordenada pelo Prof. Ricardo Anido.
   \item Apoio do CNPq desde o ano de 2002.\medskip
   \item Para alunos de primeiro e segundo grau.
   \item Possui duas modalidades:
    \begin{itemize}
     \item Iniciação: para alunos até a oitava série do ensino fundamental.
     \item Programação: para alunos que cursam até o último ano do ensino médio.
    \end{itemize}
   \item Alunos que terminaram o segundo grau no ano anterior também podem participar da Olimpíada.
  \end{itemize}
 \end{block}
\end{frame}
%- - - - - - - - - - - - - - - - - - - - - - - - - - - - - - - - - SLIDE -
\begin{frame}
 \frametitle{Olimpíada de Informática}
 \begin{block}{Como funciona?}
  \begin{itemize}
   \item Na modalidade iniciação: provas de lógica em múltipla escolha para motivar o gosto pela ciência da computação.
   \item Você deve se inscrever em alguma instituição que vá aplicar a prova (escola ou
universidade).
   \item Na modalidade Programação você deve saber programar em C ou C++.
   \item A competição é individual.
   \item A competição é em três fases.
   \item As soluções são avaliadas a posteriori.
   \item Quanto mais testes um programa resolve (retorna o resultado esperado), maior a pontuação do competidor.
  \end{itemize}
 \end{block}
\end{frame}
%- - - - - - - - - - - - - - - - - - - - - - - - - - - - - - - - - SLIDE -
\begin{frame}
 \frametitle{Olimpíada de Informática}
 \begin{block}{E se eu me der bem?}
  \begin{itemize}
   \item Os melhores colocados são convidados a fazer um curso em Campinas.
   \item Dentre os participantes do curso serão escolhidos os integrantes da equipe
brasileira na Olimpíada Internacional de Informática (IOI).
  \end{itemize}
 \end{block}
\end{frame}
%- - - - - - - - - - - - - - - - - - - - - - - - - - - - - - - - - SLIDE -
% \begin{frame}
%  \frametitle{Olimpíada de Informática}
%  \begin{block}{Como Eu Me Preparo?}
%   \begin{itemize}
%    \item Os conceitos e as técnicas de programação serão estudados com base a linguagem de programação Objective CAML.
%     \begin{itemize}
%      \item Oportunamente, e para fins comparativos, poderão ser ilustrados alguns conceitos em uma segunda linguagem de programação, como C++.
%     \end{itemize}
%    \item O conteúdo da disciplina é incremental: os conceitos mais avançados apenas podem ser entendidos quando os conceitos básicos foram bem assimilados.
%    \item A disciplina DIM0424 fornecerá aos alunos oportunidade de exercitar de forma concreta os conceitos apresentados em sala de aula.
%    \item Atenção: o conteúdo da disciplina é abrangente e necessita um esforço importante e permanente dos alunos.
%    \item Os alunos recebem tarefas a serem realizadas antes da aula seguinte.
%   \end{itemize}
%  \end{block}
% \end{frame}
% %- - - - - - - - - - - - - - - - - - - - - - - - - - - - - - - - - SLIDE -
